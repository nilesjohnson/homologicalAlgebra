\documentclass[nooutcomes]{ximera}
\title{Resolutions}
\begin{document}
\begin{abstract}
  Replace a chain complex with an equivalent one that is better-behaved.
\end{abstract}
\maketitle



\subsection{email to Brian Boe}

We often introduce Ext and Tor by observing that Hom and Tensor don't preserve exact sequences.  But why should that be something we regard as "broken" and which we need to "fix"?  And why on Earth should resolutions be the fix?!

The fact that Hom/Tensor don't preserve exact sequences is equivalent to the fact that they are not homology-invariant.  It can happen that we have two chain complexes $M$ and $M'$, with a homology isomorphism

\[
M \to M'
\]

but for other chain complexes $N$, the induced map

\[
Hom(M',N) \to Hom(M,N)
\]

is \emph{not} a homology isomorphism.  In particular, $M'$ could have trivial homology, but $Hom(M',N)$ might not.

The failure of $Hom$ to preserve homology isomorphisms means that one cannot naively construct a well-defined category where the homology isomorphisms are actual isomorphisms (one would have isomorphic objects, but the morphisms out of them would not be isomorphic -- this is a contradiciton).  The derived category accomplishes this, and the goal is to motivate this from the perspective of wanting to study homology isomorphisms.

The same problem happens in topology: For example, the Warsaw Circle,
$W$, has trivial homotopy groups, but $Hom(W, S^1)$ has non-trivial
homotopy groups (it's $\pi_0$ is the integers, just like $Hom(S^1,
S^1)$).  In topology, this problem is resolved by replacing a space by
a CW complex.  This problem of $X \to X'$ being a homotopy isomorphism (that is, isomorphism on homotopy groups) but $Hom(X',Y) \to Hom(X,Y)$ not being a homotopy isomorphism \emph{never} happens when $X$ and $X'$ are CW complexes (that's the Whitehead theorem, and it's all the topology you need to know for this).

In homological algebra, the analog of a CW complex is a levelwise
projective complex.  A projective resolution $P$ of a module $M$ is
precisely a levelwise projective complex with a homology isomorphism
$P \to M$.

The analog of Whitehead's theorem for chain complexes is the result
that a homology isomorphism between levelwise projective complexes has
a chain-homotopy inverse.  I think it's usually not named, but it's
the technical result necessary to prove that Ext does not depend on
the choice of projective resolution.


Back to spaces for a minute: every topological space is
homotopy-isomorphic to a CW complex, so one can take the subcategory
consisting only of CW complexes, and obtain a well-defined category
from this by inverting the weak equivalences -- indeed, because the
homotopy-isomorphisms in this subcategory are homotopy equivalences,
the 'derived' category can be formed by simply taking homotopy
equivalence classes of maps.  This produces a well-defined category
where the homotopy-isomorphisms are actually isomorphisms.
Constructions such as "the" homotopy fiber of an arbitrary map can be
understood in categorical terms here, simplifying arguments which
would normally involve a complicated mess of homotopies to check.

The derived category of a ring has the same advantage!  There is one
slight additional subtlety for homological algebra though: the notion
of \emph{injectivity}.  Spaces don't have this subtlety...





\subsection{Motivation}

There are pathologies in the category of chain complexes.  For
example, $Hom_{\mathbb{Z}}(C, \mathbb{Z})$ could have nontrivial
homology even if $C$ has trivial homology!  Same problem, e.g., for $-
\otimes \mathbb{Z}/p$.

This means that a category of chain complexes modulo weak equivalence
wouldn't have well-defined hom objects or tensor products.

This kind of problem occurs for topological spaces too:  The Warsaw
circle, $W$, has trivial homotopy groups, but the mapping space from
$W$ to the circle has nontrivial homotopy!
\[
[picture]
\]






\subsection{Solution for spaces}

In spaces, the CW complexes form a well-behaved subcategory, and every
space is weakly equivalent to a CW complex.  So one could ``restrict''
to CW complexes.  We still do want to consider all spaces, so develop
a system for coherently replacing any space with a CW complex . . .

The analogous solution in chain complexes comes in the form of
resolutions!  This can be done in two different (dual) ways . . .

\subsection{Projective resolutions}



\subsection{Injective resolutions}



\end{document}
