\documentclass{ximera}
\title{Introduction}
\begin{document}
\begin{abstract}
Overview of the course.
\end{abstract}
\maketitle

\section{What is homological algebra?}

Homological algebra is a place to get answers. All kinds of subtle and
interesting answers! Homological algebra is a collection of tools and
techniques which are used in any field with algebra in its name:
Algebra, algebraic topology, algebraic geometry, algebraic number
theory, etc.

With homological algebra, we can reduce difficult questions about
complex objects to basic linear algebra problems. Albeit an infinite
sequence of problems, but basic problems nonetheless! One might
compare this with the way an analytic function can be understood
entirely by its Taylor coefficients: Derivatives are easy, and you can
understand something subtle (an arbitrary function) by doing an
infinite sequence of easy computations.

In practice, this means that the sheer magnitude of objects in
homological algebra can be overwhelming to the novice. In this course
we'll give an organized introduction and overview of the main
ideas. We'll work through them in some of the classic—and most
useful—applications, and we'll introduce enough special topics to
pique the interest of students from a variety of backgrounds.



\section{Outline}

Our text will be Weibel's
\href{http://www.amazon.com/gp/offer-listing/0521559871/ref=dp_olp_used?ie=UTF8&condition=used}{An
  Introduction to Homological Algebra}, and most of the course will
follow this text. We'll cover the basic concepts of homological
algebra with most of our attention focused on central applications. At
relevant points in the course, we'll foray into related topics which
are of interest to the students.

Where there are expositional choices to be made, we tend toward
topological and categorical descriptions as a conceptual framework for
the techniques of homological algebra. Background in these areas may
be helpful, but is not required. The only true prerequisite for the
course is familiarity with abelian groups and quotients thereof. More
general familiarity with modules over commutative rings,
homomorphisms, and tensor products will also be useful.

A potential syllabus is given below, although the pacing and selection
of additional topics will be revised depending on the audience.

\subsection{The concepts}

\begin{itemize}
\item Chain complexes and homology
\item Derived functors and derived categories
\item Spectral sequences
\end{itemize}

\subsection{The applications}

\begin{itemize}
\item Homology and cohomology of spaces and of finite groups
\item Ext and Tor
\item The Serre spectral sequence and spectral sequences arising from exact couples
\end{itemize}

\subsection{The additional topics}

\begin{itemize}
\item Homological dimension
\item Lie algebra co/homology
\item Hochschild co/homology
\item Sheaf cohomology
\item Model categories and derived categories
\end{itemize}



\subsection{Course status}

This Ximera course is based on a course taught to graduate students at
the University of Georgia in Spring 2012.  Each ``activity'' here
represents one day's lecture, and the exercises are those assigned in
class.

At present, not all lectures have been transcribed to this Ximera
course, and the transcription progress is not linear.  Below is the
list of activities, together with their completion status.

\begin{enumerate}
\item (not done) \href{/activity/chainComplexes/chainComplexes/}{Chain complexes}
\item (not done) \href{/activity/lesInHomology/lesInHomology/}{LES in homology}
\item (not done) \href{/activity/chainHomotopy/chainHomotopy/}{Chain homotopy}
\item (not done) \href{/activity/mappingConesAndCylinders/mappingConesAndCylinders/}{Mapping Cones and Cylinders}
\item (not done) \href{/activity/resolutions/resolutions/}{Resolutions}
\item (not done) \href{/activity/derivedFunctors/derivedFunctors/}{Derived functors}
\item (not done) \href{/activity/exactFunctors/exactFunctors/}{Exact functors}
\item (not done) \href{/activity/categoriesFunctorsAndNaturalTransformations/categoriesFunctorsAndNaturalTransformations/}{Categories, functors, and natural transformations}
\item (not done) \href{/activity/adjunctions/adjunctions/}{Adjunctions}
\item (not done) \href{/activity/adjunctionsAndTheYonedaLemma/adjunctionsAndTheYonedaLemma/}{Adjunctions and the Yoneda lemma}
\item (not done) \href{/activity/adjunctionsAndExactness/adjunctionsAndExactness/}{Adjunctions and exactness}
\item (not done) \href{/activity/balancingTorAndExt/balancingTorAndExt/}{Balancing Tor and Ext}
\item (not done) \href{/activity/universalCoefficientTheorem/universalCoefficientTheorem/}{Universal coefficient theorem}
\item (not done) \href{/activity/homologicalDimension/homologicalDimension/}{Homological dimension}
\item (not done) \href{/activity/localRings/localRings/}{Local rings}
\item (not done) \href{/activity/koszulComplexes/koszulComplexes/}{Koszul complexes}
\item (not done) \href{/activity/gorensteinRings/gorensteinRings/}{Gorenstein rings}
\item (not done) \href{/activity/groupCohomology/groupCohomology/}{Group cohomology}
\item (not done) \href{/activity/localCohomology/localCohomology/}{Local cohomology}
\end{enumerate}



\begin{exercise}
  What is homological algebra useful for?
  Choose the best answer.
  \begin{multipleChoice}
    \choice{Getting concrete answers to subtle questions}
    \choice{Understanding higher-order structure}
    \choice{Proving theorems}
    \choice[correct]{All of these are good answers}
  \end{multipleChoice}
\end{exercise}

\end{document}
