\documentclass[nooutcomes]{ximera}
\title{Chain homotopy}
\begin{document}
\begin{abstract}
  An important equivalence relation on chain maps
\end{abstract}
\maketitle

The notion of chain homotopy can be explained by an "interval" chain
complex, I, and map

\[
M \otimes I \to N
\]

Suppose $R$ is the ground ring.  The chain complex $I$ is nontrivial in two degrees:

\begin{align*}
I_0 & = R \oplus R\\
I_1 & = R
\end{align*}

Suppose the two generators for $I_0$ are $a$ and $b$, and suppose the
generator for $I_1$ is $c$.  Then the differential is given by

\[
d(c) = b - a.
\]

It's a fun exercise to check that the usual definition of chain
homotopy equivalence is the same as the one determined by a map out of
$M \otimes I$.  (The part that shifts degree by 1 is the part coming
from $I_1$).

In fact, this $I$ is the chain complex you would get from the cellular
chains on the standard interval $I$, so from that point of view it's
not so surprising that it plays such a crucial role.




\end{document}
