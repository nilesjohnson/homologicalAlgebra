\documentclass{ximera}
\title{Spectral sequences: introduction}
\begin{document}
\begin{abstract}
  Filtration on a chain complex leads to filtration on its homology.
\end{abstract}
\maketitle

\section{Rough idea}

We will be interested in a (first-quadrant) double-complex $E_{*,*}$
and the homology of its associated total complex $T_* = \bigoplus_{p +
  q = *} E_{p,q}$.

(Double complex means there are ``vertical'' and ``horizontal'' differentials $d^v$ and $d^h$ with the skew-commuting relation $d^h d^v + d^v d^h = 0$.)

\begin{image}
  (insert picture of first-quadrant double complex)
\end{image}

\begin{enumerate}
\item Let $E^0_{p,q}$ be the sequence of vertical complexes $(E_{p,*},d^v)$ for each $p$.
\item Let $E^1_{p,q} = H_q(E^0_{p,*}, d^v)$.  Then $d^h$ induces
  \[
  d^1: E^1_{p,q} \rightarrow E^1_{p-1,q}.
  \]
\item Let $E^2_{p,q} = H_p(E^1_{*,q}, d^1)$.
\item There is an induced differential on $E^2$ of bidegree $(-2,1)$,
  and it's homology gives a bigraded complex $E^3$.
\item Each $E^r$ is the $r-th$ page of the spectral sequence, and it's
  homology gives the $(r+1)-st$ page.
\item Successive pages give successive approximations to the homology
  of the total complex $T$.
\end{enumerate}


We discuss spectral sequences arising from filtered chain complexes.
This is not the most general point of view, but is broadly applicable
and reasonably natural.

\section{Special case}

If $E$ is concentrated in just two adjacent columns, $p-1$ and $p$,
there is a short exact sequence
\[
0 \rightarrow E^2_{p-1,q+1} \rightarrow H_{p+q} \rightarrow E^2_{p,q} \rightarrow 0.
\]


\begin{note}
It can be easier to draw spectral sequences using Adams grading . . .
\end{note}

\end{document}
